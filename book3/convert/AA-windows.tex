% The contents of this file is 
% Copyright (c) 2009-  Charles R. Severance, All Righs Reserved

\chapter{Python Programming on Windows}

In this appendix, we walk through a series of steps
so you can run Python on Windows.  There are many different 
approaches you can take, and this is just one
approach to keep things simple.

First, you need to install a programmer editor.  You
do not want to use Notepad or Microsoft Word to edit
Python programs.  Programs must be in "flat-text" files
and so you need an editor that is good at
editing text files.

Our recommended editor for Windows is NotePad++ which
can be downloaded and installed from:

\url{https://notepad-plus-plus.org/}

Then download a recent version of Python 2 from the
\url{www.python.org} web site.

\url{https://www.python.org/downloads/}

Once you have installed Python, you should have a new
folder on your computer like {\tt C:{\textbackslash}Python27}.

To create a Python program, run NotePad++ from the Start Menu
and save the file with a suffix of ``.py''.  For this
exercise, put a folder on your Desktop named 
{\tt py4inf}.  It is best to keep your folder names short
and not to have any spaces in your folder or file name.

Let's make our first Python program be:

\beforeverb
\begin{verbatim}
print 'Hello Chuck'
\end{verbatim}
\afterverb
%
Except that you should change it to be your name.  Save the file
into {\tt Desktop{\textbackslash}py4inf{\textbackslash}prog1.py}.

Then open a command-line window.  Different versions of Windows
do this differently:

\begin{itemize}
\item Windows Vista and Windows 7: Press {\bf Start}
and then in the command search window enter the word
{\tt command} and press enter.

\item Windows XP: Press {\bf Start}, then {\bf Run}, and 
then enter {\tt cmd} in the dialog box and press {\bf OK}.
\end{itemize}

You will find yourself in a text window with a prompt that
tells you what folder you are currently ``in''.  

Windows Vista and Windows-7: {\tt C:{\textbackslash}Users{\textbackslash}csev}\\
Windows XP: {\tt C:{\textbackslash}Documents and Settings{\textbackslash}csev}

This is your ``home directory''.  Now we need to move into 
the folder where you have saved your Python program using
the following commands:

\beforeverb
\begin{verbatim}
C:\Users\csev\> cd Desktop
C:\Users\csev\Desktop> cd py4inf
\end{verbatim}
\afterverb
%
Then type 

\beforeverb
\begin{verbatim}
C:\Users\csev\Desktop\py4inf> dir 
\end{verbatim}
\afterverb
%
to list your files.  You should see the {\tt prog1.py} when 
you type the {\tt dir} command.

To run your program, simply type the name of your file at the 
command prompt and press enter.

\beforeverb
\begin{verbatim}
C:\Users\csev\Desktop\py4inf> prog1.py
Hello Chuck
C:\Users\csev\Desktop\py4inf> 
\end{verbatim}
\afterverb
%
You can edit the file in NotePad++, save it, and then switch back
to the command line and execute the program again by typing
the file name again at the command-line prompt.

If you get confused in the command-line window, just close it
and open a new one.

Hint: You can also press the ``up arrow'' at the command line to 
scroll back and run a previously entered command again.

You should also look in the preferences for NotePad++ and set it 
to expand tab characters to be four spaces.  This will save you lots
of effort looking for indentation errors.

You can also find further information on editing and running 
Python programs at \url{www.py4inf.com}.

